\documentclass[12pt]{article}

\title{Results and interpretation for practical: "Is Florida getting warmer?"}

\author{Chuxinyao Wang}

\date{}

\begin{document}
  \maketitle
  
  \begin{results}
    By shuffling the temperature for 10 times and record every correlation cofficient, here are 10 different correlation cofficient:  
    \\ \hspace*{\fill} \\
    [1] -0.215896250 [2]-0.091929733 [3]-0.065370498 [4]-0.001536993 [5]-0.004206969
    [6] -0.041914525 [7] 0.111322190 [8] 0.145428792 [9]-0.046036593 [10]-0.094968587
    \\ \hspace*{\fill} \\
    And the original correlation cofficient for 'Temperature' and 'Year' is: 
    \\ \hspace*{\fill} \\
    [1] 0.5331784
    \\ \hspace*{\fill} \\
    It can be easily seen that all correlation cofficient after shuffled are smaller than original one. This can be explained that the temperature does not belong to random distribution, 
    it does has some relationship with Year instead of accidental occassion. 'Temperature' and 'Year' are related to some extend.
  \end{results}
 

\end{document}